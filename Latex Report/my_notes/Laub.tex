\documentclass[11pt]{book}
\input{configuration} % various packages needed for maths etc.



\begin{document}
\frontmatter
\pagestyle{front}

\mainmatter
\pagestyle{main}





























\section{birth immigration process}

\begin{theoreme}{A}
$$0 < n   := \int_0^{\infty} \mu (s) ds < 1  $$ 

and 

$$\int_0^{\infty} s \mu(s) ds < \infty $$

then the number of HP arrivals in $[0,t]$ is asymptotically normally distributed.
\end{theoreme}

We say that the direct offspring of the natural process are called the first-generation, and their offspring belong to the second-generation and so on. The union of all generations bigger than $1$ are naturally called the descendants.

We can compute the branching ratio. It is defined as the $L^1$ norm of the underlying process and in the case of an exponentially decaying intensity we get:

$$ n = \frac{\alpha}{\beta} $$

It is possible to inspire from the immigration-birth representation in order to simulate the process.



\section{stationarity}

the term `stationary' has many different meanings in probability theory. In this context the HP is stationary when the jump process $(dN(t) : t \geq 0)$, which takes values in $\{0; 1\}$, is
weakly stationary. This means that $\E[dN(t)]$ and $\Cov(dN(t)$; $dN(t+s))$ do not depend on $t$. Stationarity
in this sense does not imply stationarity of $N(\cdot)$ or stationarity of the inter-arrival times [25]. One
consequence of stationarity is that $\lambda^* ( \cdot )$ will have a long term mean.
25 is P.A. Lewis, Journal of Sound and Vibration 12(3), 353 (1970)


Asmussen in 24. S. Asmussen, Applied Probability and Queues, 2nd edn. Applications of Mathematics: Stochastic Modelling and
Applied Probability (Springer, 2003) is showing $$\E [ \lambda(t) ] \to_{t \to \infty } \frac{\nu}{1-n}$$

and when $n < 1$, thr autocovariance defined as $$ R( \tau ) = \Cov \left ( \frac{dN(t)}{dt}, \frac{dN(t)}{dt} \right )= \frac{\alpha \beta \lambda ( 2 \beta - \alpha ) }{2 ( \beta - \alpha ) ^2 }  e^{- ( \beta - alpha ) \tau }  $$



\section{markov}

Apparently $(N(t), \lambda(t \mid \mathcal F_{t^-} ))$ is markovian, cf the phd thesis. 

\cleardoublepage% le corps du document est terminé
\appendix
\pagestyle{back}





\backmatter
\end{document}