\documentclass[11pt]{book}
\input{configuration} % various packages needed for maths etc.



\begin{document}
\frontmatter
\pagestyle{front}

\mainmatter
\pagestyle{main}

\section{cox}
Many applications give rise to point patterns exhibiting clustering. For example point patterns of positions of trees and plants often show clustering, either because plants tend to emerge where the soil has particular properties or because of seed setting mechanisms. Such clustered point patterns have traditionally been modelled by Cox processes or cluster point processes. A classical example concerns the spatial distribution of insect larvae, typically exhibiting clustering around the egg masses from which the larvae are hatched [10]. Neyman and Scott [11] introduced the use of cluster processes in the study of clusters in locations of galaxies. Cox processes [6] are Poisson processes with a random intensity and are natural models to consider when clustering is due to spatially varying features such as soil properties. 

Cluster point processes are generated by first constructing a point process of unobserved (perhaps marked) parent points, each of which give rise to a random number of observed daughter points. The process of daughter points is the cluster process; it arises naturally as a model for point patterns where clustering is due to mechanisms such as seed setting.

the paper must be : cox







\section{phd thesis:}
There is also a dierent way of looking
at a Hawkes process. If one ignores the location of the event, i.e. if one ignores
the time dimension, a Hawkes process is simply a branching process. In other
words, the events are related to each other in the way as ancestors and ospring
are related to each other. As a mental picture, one can think of one or several
trees, where each node corresponds to one of the events. The root nodes
corresponds to immigrants and the branches correspond to the descendants of
immigrants




Simulation of Stationary Hawkes Process. Often one wants to simulate
the stationary version of the process in a nite time window. Unfortunately,
the standard method for the simulation method described above does not work
in this case, as the past of the process is not know and cannot be simulated,
at least not completely.
If one ignores the past of the process and simply starts to simulate the
process at some given time, one speaks about an approximate simulation. In
this case, one is actually simulating a transient version and not the stationary
version of the process. But if one simulates for a long enough time interval, the
so-called burn-in period, then the transient version converges to the stationary
one. This is a consequence of the stability properties, which are described below.
Since the deviation between the transient and stationary process becomes
negligible, this simulation method is good enough for practical applications.

Perfect Simulation. A simulation method which directly simulates the stationary
version without approximation is a so-called perfect simulation method.
The idea is to incorporate somehow the eect of past observations without actually
simulating the past of the process. In the context of point processes, this
type of simulation method has rst been described in [BK02].




As explained, one can dene a Hawkes process
either as a recursive Poisson cluster process, or one can dene a Hawkes process
by specifying its intensity function. In this introduction, I will only look at the
second representation. The reason is that in order to calculate the likelihood
function one needs the intensity process, and therefore this is the suitable
representation if one wants to t a Hawkes model to a set of empirical data.



Impact Function. The amount by which the intensity increases after an
event does not only depend on the time lag but also on the mark value of
the triggering event. The in
uence of the mark value is governed by the impact
functions. The impact functions describe again only the relative eect of
an event, since they appear as a multiplicative factor in the denition of the
intensity function.

Denition (Immigration intensity). Let d  1 be the number of components.
Depending on the version of multivariate Hawkes process, assume the
following constants are given

\section{compensator}
Likelihood Function. The standard way to estimate the parameters of a
Hawkes process model is to maximize the likelihood function. In order to
dene the likelihood function, one rst has to x an observation period D, i.e.
the time interval during which the data has been collected. I will always use
the observation period D := [T; T], for xed times T < T.
The compensator will be useful for the calculation of the likelihood function.
Recall Denition 1.6 of the cumulative decay functions  wj and  w:
1.17 Denition (Compensator). For all t 2 D dene in the genuine and
pseudo multivariate case:
j(t) :=
Z t
T
j(s)ds and (t) :=
Z t
T
(s)ds:
This is the general denition. For a Hawkes process one obtains:

p39


Burn-In Period. For simplicity, the algorithm I present below starts at time
t0 := 0 and assumes an empty initial state, see also Denition 6.34. This
means the algorithm starts with no points dened initially; but one could easily
change that. If the regularity conditions from Theorem 1.16 are satised, then
the simulated Hawkes process converges in a strong sense to an associated
stationary Hawkes process. The details of this convergence are explained in
Proposition 6.52, and especially Theorem 6.55 and its proof.
This implies that the rst few events generated by this simulation algorithm
should not be used. In other words, one has to let the algorithm run for some
time, the so-called burn-in period, before one uses the generated events.



\cleardoublepage% le corps du document est terminé
\appendix
\pagestyle{back}





\backmatter
\end{document}